\documentclass[aspectratio=169]{beamer}
\usepackage{xeCJK}
\usepackage{graphicx}
\usepackage{listings}
\setCJKmainfont[Path = /usr/share/fonts/noto/, Extension = .otf]{NotoSansCJKtc-Regular}
\setCJKmonofont[Path = /usr/share/fonts/noto/, Extension = .otf]{NotoSansMonoCJKtc-Regular}

\XeTeXlinebreaklocale "zh"
\XeTeXlinebreakskip = 0pt plus 1pt

\usetheme{Singapore}

\begin{document}
\author{曾志宏}
\title{\LaTeX 和 cmake}
\date{\today}

\begin{frame}
\maketitle
\end{frame}

\begin{frame}
什麼是 \LaTeX ?
\begin{itemize}
\item Well...你知道的
\end{itemize}

\bigskip

什麼是cmake\cite{cmake}?
\begin{itemize}
\item 跨平台的 meta-build 的軟體, 可自動生成 Makefile
\item 編譯 \LaTeX 文件只需下一個 make 指令
\item 支援 out-of-source build, 生成的檔案放在獨立的目錄裡, 不影響原始目錄
\end{itemize}
\end{frame}

\begin{frame}
cmake 用到的檔案:
\begin{itemize}
\item CMakeLists.txt --- cmake 的規則檔, 視文件的需求修改
\item UseLATEX.cmake\cite{UseLATEX} --- 直接使用
\end{itemize}
\end{frame}

\begin{frame}[fragile]{CMakeLists.txt的內容}
\lstinputlisting{CMakeLists.txt}

INPUTS 變數裡儲存編譯時會使用到的檔案, 如圖檔, 程式碼等等
\end{frame}

\begin{frame}[fragile]{編譯指令}
以本文件為例:

\lstinputlisting[language=bash]{go.sh}

生成的文件在 out 目錄裡
\end{frame}

\begin{frame}
\Huge That's it.
\end{frame}

\bibliography{ref}
\bibliographystyle{plain}
\end{document}
